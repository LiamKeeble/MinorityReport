\documentclass[a0paper,fleqn]{betterposter}

\begin{document}	
\betterposter{
%%%%%%%% MAIN COLUMN

\maincolumn{
%%%% Main space

	\textbf{Minority Report}
\\Main argument/finding \textbf{etc.}

}{
%%%% Bottom space

}

}{
%%%%%%%% LEFT COLUMN

\title{Are there geographical biases in the study of language variation and change?}
\author{Caitlin Halfacre}
\author{Liam Keeble}

\institution{Newcastle University}

\section{Introduction}

Bias in science is unavoidable, and sociolinguistics is no different. In order to combat sources of bias, we must first identify its existence. With this study we aim to:
\begin{itemize}
\item Assess whether there is bias towards studying some varieties of native English over others.
\item Assess the effect of geographical locations of research institutions on the frequency of studies on different varieties of native English.
\item To assess whether certain other characteristics of research and geography (e.g. the existence of corpora/average income of the area where the variety is spoken) of varieties effects the frequency of studies published.
\end{itemize}

}
{
%%%%%%%%%Right column
	
\section{Methods}

In order to meet our aims we:
\begin{itemize}
	\item Systematically searched for studies on each particular variety of native English as identified by wikipedia.
	\item Wikipedia was used so as it serves as a kind of public self-identifier for speakers of speaking different varieties.
	\item The data returned from searches was manually assessed for relevance to our study, and removed if it was not relevant.
	\item The remaining studies from each search were counted and the frequency of papers found per variety were used the outcome variable in generalised linear models.
	\item Data for predictor variables was found using systematic google searches, and included the existence of a corpus for a certain variety, the average income of the area where the variety is spoken, and the geographical distance between the local area of a variety and the nearest university.

\end{itemize}



\section{Pilot results}



}
\end{document}



