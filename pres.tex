
\documentclass{beamer}
\usetheme{Montpellier}
\title{Factors affecting bias towards researching varieties of English}
\author{Caitlin Halfacre and Liam Keeble}
\date{2021}
\begin{document}


\frame{\titlepage}


\section{Problems to be addressed}
\begin{frame}
\frametitle{General problems}
\begin{itemize}
\item Linguistics is a young science, so although bias may be limited, it is important to ensure high quality research practice in order to avoid making the mistakes of other fields.
\item It is important therefore to identify and minimise any forms of bias including publication bias, researcher bias and any forms of bias more specific to the field at hand.
\item Only then will the most representative data be obtainable.
\end{itemize}
\end{frame}


\begin{frame}
\frametitle{Specific problems}
\begin{itemize}
\item Language variation and change is a field where further forms of bias than those universal to all scientific fields can enter.
\item When trying to find universal principles of change, it is necessary to establish whether certain changes are consistent across different varieties of a language.
\item To do that, researchers need as broad a dataset of varieties as possible.
\item However, if there is bias towards researching only specific varieties, then such questions cannot be asked.

\end{itemize}

\end{frame}

\section{Research questions}
\begin{frame}
\frametitle{Research questions}
\begin{enumerate} 
		\item Are English varieties that are geographically distant from linguistics university departments more likely to be understudied? \label{geog}
		
		\item Does the presence of a locally focused corpus increase the research output on a particular variety?  \label{corpus}
		
		\item Are English varieties associated with higher social/income status lacking in research articles? \label{income}
		
		\item Is more research conducted on varieties of English associated with suburban, metropolitan, or rural areas? \label{suburban}
	\end{enumerate}
\end{frame}


\section{Methods}
\begin{frame}
\frametitle{Data collections}
\begin{itemize}
\item A systematic literature search to quantify frequency of publications on a specific variety.
\item Geographical measurements using R and Google Maps.
\item Systematic online searches to establish the existence of corpus (corpi?) and the average income of areas of interest.
\end{itemize}
\end{frame}

\begin{frame}
\frametitle{Data analysis}
\begin{itemize}
\item Each dependent variable will be assessed as a predictor of frequency of publications for different varieties of English.
\item Details are available on the Open Science Framework: https://osf.io/h7t6x/
\end{itemize}
\end{frame}

\begin{frame}
\frametitle{Thank you for listening}
Any questions?
\end{frame}



\end{document}