
\documentclass{beamer}
\usetheme{Montpellier}
\usecolortheme{albatross}


\title{Factors affecting bias towards researching varieties of English}
\author{Caitlin Halfacre and Liam Keeble}
\date{2021}
\begin{document}


\frame{\titlepage}


\section{Problems to be addressed}
\begin{frame}
%Liam
\frametitle{General problems}
\begin{itemize}
\item We know that bias in science exists
\item Publication bias, researcher bias, selection bias...
\item Linguistics is a young science, and although this means bias may be limited, it likely still exists.
\item However, there may be more biases in certain fields that are more specific to those fields.
\end{itemize}
\end{frame}


\begin{frame}
%Caitlin
\frametitle{Specific problems}
\begin{itemize}
\item LVC is a field where further forms of bias than those universal to all scientific fields can enter.
\item Are we studying language varieties based on something unconscious? \\ e.g. how near to a university they are.
\item bias in what we choose to study leads to gaps in the broader LVC data
\end{itemize}
\end{frame}

\begin{frame}
\frametitle{Specific Consequences}
%Caitlin
\begin{itemize}
\item If we have systematic gaps how can we be certain of broader conclusions we draw about principles of variation and change?
\item It is important therefore to identify and minimise any forms of bias including publication bias, researcher bias and any forms of bias more specific to the field at hand.
\item Only then will the most representative data be obtainable.
\end{itemize}
\end{frame}

\section{Research questions}
\begin{frame}
\frametitle{Research questions}
%Liam
How can we go about assessing possible bias in Language Variation and Change Research? A few possible questions.
\begin{enumerate} 
		\item Are English varieties that are geographically distant from linguistics university departments more likely to be understudied? \label{geog}
		
		\item Does the presence of a locally focused corpus increase the research output on a particular variety?  \label{corpus}
		
		\item Are English varieties associated with higher social/income status lacking in research articles? \label{income}
		
		\item Is more research conducted on varieties of English associated with suburban, metropolitan, or rural areas? \label{suburban}
\end{enumerate}
\end{frame}


\section{Methods}
\begin{frame}
%Liam
\frametitle{Data collection}
\begin{itemize}
\item A systematic literature search to quantify frequency of publications on a specific variety.
\item Geographical measurements using R and Google Maps.
\item Systematic online searches to establish the existence of corpora and the average income of areas of interest.
\end{itemize}
\end{frame}

\begin{frame}
%Liam
\frametitle{Data analysis}
\begin{itemize}
\item Each dependent variable will be assessed as a predictor of frequency of publications for different varieties of English.
\item Details are available on the Open Science Framework: https://osf.io/h7t6x/
\end{itemize}
\end{frame}

\begin{frame}
\frametitle{Thank you for listening}
%Caitlin
Any questions?

\only<2>{
Our questions for you:
\begin{itemize}
\item Are you convinced by the possibility and consequences of bias in LVC research?
\item Any tips on submitting this as as a \textbf{methods} paper to a conference
\end{itemize}
				}
\end{frame}

\end{document}