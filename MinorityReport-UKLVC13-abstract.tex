\documentclass[12pt,a4paper]{article}

\usepackage{hyperref}
\usepackage{float}
\usepackage[round]{natbib} %citation package
	\bibliographystyle{agsm}
\usepackage[dvipsnames]{xcolor} %general colours
\usepackage{todonotes} 
\usepackage[margin=1in]{geometry}

\hypersetup{
	draft=true,
	final=true,
	colorlinks=true,
	linktoc=all,
	linkcolor=MidnightBlue,
	citecolor=blue,
	urlcolor=red
	%	allcolors=black
}

\renewcommand{\familydefault}{\sfdefault}


\newcommand{\CHcomment}[1]{\todo[color=magenta]{#1}}
\newcommand{\LKcomment}[1]{\todo[color=cornflowerBlue]{#1}}
\newcommand{\todowording}[1]{\todo[color=Aquamarine]{#1}}
\newcommand{\todostructure}[1]{\todo[color=Thistle]{#1}}
\newcommand{\todoreference}[1]{\todo[color=SeaGreen]{#1}}
\newcommand{\todocontent}[1]{\todo[color=RoyalPurple]{#1}}
\newcommand{\todocontentinline}[1]{\todo[color=RoyalPurple,inline]{#1}}

\title{Assessing research bias against English varieties: a systematic review}
\author{Caitlin Halfacre$^{2}$ and Liam Keeble$^{1}$}
\date{$^{1}$Henry Wellcome Building, Medical School, Newcastle Upon Tyne, NE2 4HH, United Kingdom\\
$^{2}$Percy Building, School of English Literature, Language and Linguistics, Newcastle University, Newcastle upon Tyne, NE1 7RU, United Kingdom\\}


\begin{document}
\begin{center}
	\textbf{Minority Report}\\
\end{center}

\todocontentinline{Current word count 297}
There is bias in all science. Some forms of bias are universal across disciplines, and some are specific problems for specific disciplines. Here we propose an attempt to assess some sources of bias which may be affecting studies in the field of language variation and change (LVC).  
A consequence of such bias is that the literature on variation and change tends to be clustered around particular varieties \cite{Trudgill2002}. Bias like this can lead to gaps in the data and hence casts doubt on broader conclusions drawn about the principles of variation and change. This paper sets out a novel methodology aiming to understand some of the possible biases present in our approach to researching specifically varieties of English spoken in England. 
	
The proposed research question is: which of the following factors affect a variety's likelihood of being studied?
	\begin{itemize}
		\item geographical distance from a university with a Linguistics department
		\item presence of a locally focussed corpus
		\item association with higher or lower social/income status
		\item whether the area it is associated with is suburban, metropolitan, or rural
	\end{itemize}


To answer this question, systematic literature searches will be used to estimate the frequencies of studies on different varieties (dependent variable). Independent variables will be measured using systematic online searches and online geographical measurements. Studies returned from searches will be assessed for their relevance, and only studies published between 1982 \citep{Wells1982b} and 2019, and studies studying sociophonetic change in apparent time, will be included.

We are aware that even in these questions we are focussing on a very small subset of LVC research, but present this as a methods paper hoping to open a discussion of bias in the field in general, and that others will adapt our methods for studies of bias in other related research.

\bibliography{MinorityReferences.bib}

\end{document}
