\documentclass[12pt,a4paper]{article}

\usepackage{hyperref}
\usepackage{float}
\usepackage[round]{natbib} %citation package
	\bibliographystyle{agsm}
\usepackage[dvipsnames]{xcolor} %general colours
\usepackage{todonotes} 
\usepackage[margin=1in]{geometry}

\hypersetup{
	draft=true,
	final=true,
	colorlinks=true,
	linktoc=all,
	linkcolor=MidnightBlue,
	citecolor=blue,
	urlcolor=red
	%	allcolors=black
}

\renewcommand{\familydefault}{\sfdefault}


\newcommand{\CHcomment}[1]{\todo[color=magenta]{#1}}
\newcommand{\LKcomment}[1]{\todo[color=cornflowerBlue]{#1}}
\newcommand{\todowording}[1]{\todo[color=Aquamarine]{#1}}
\newcommand{\todostructure}[1]{\todo[color=Thistle]{#1}}
\newcommand{\todoreference}[1]{\todo[color=SeaGreen]{#1}}
\newcommand{\todocontent}[1]{\todo[color=RoyalPurple]{#1}}
\newcommand{\todocontentinline}[1]{\todo[color=RoyalPurple,inline]{#1}}

\title{Assessing research bias against English varieties: a systematic review}
\author{Caitlin Halfacre$^{2}$ and Liam Keeble$^{1}$}
\date{$^{1}$Henry Wellcome Building, Medical School, Newcastle Upon Tyne, NE2 4HH, United Kingdom\\
$^{2}$Percy Building, School of English Literature, Language and Linguistics, Newcastle University, Newcastle upon Tyne, NE1 7RU, United Kingdom\\}


\begin{document}
\begin{center}
	\textbf{Minority Report}\\
\end{center}

\todocontentinline{Current word count 261}

In this study we attempt to assess the existence of biases arising from the geographical locations of research institutions, and their e+6ffect on empirical study in the field of language variation and change (LVC).  
A consequence of such biases is that the literature on variation and change tends to be clustered around particular varieties \cite{Trudgill2002}. Bias like this can lead to gaps in the data and hence casts doubt on broader conclusions drawn about the principles of LVC. This paper sets out a novel methodology aiming to understand some of the possible biases present in our research, beginning with varieties of English spoken in England. 
	
Our research question is: which of the following factors affect a variety's likelihood of being studied?
	\begin{itemize}
		\item geographical distance from a university with a Linguistics department
		\item presence of a locally focussed corpus
		\item association with higher or lower social/income status
		\item whether the area it is associated with is suburban, metropolitan, or rural
	\end{itemize}


To answer this question, systematic literature searches \citep{lefebvre2019searching} are used to estimate the frequencies of studies on different varieties of English (dependent variable). Independent variables are measured using systematic online searches and online geographical measurements (more methods and analysis details can be found at: https://osf.io/bp3es). Studies returned from searches are assessed for their relevance, and only studies published between 1982 \citep{Wells1982b} and 2019, and studies assessing sociophonetic change in apparent time, are included, so that we are assessing bias in studies that use fairly similar methods and theoretical approaches.

\pagebreak
\bibliography{MinorityReferences.bib}

\end{document}
